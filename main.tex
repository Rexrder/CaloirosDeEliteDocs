\documentclass[a4paper,11pt]{article}
\usepackage[utf8]{inputenc}
\usepackage{microtype} %improves justification
\usepackage{graphicx}
\usepackage{wrapfig}
\usepackage{enumitem}
\usepackage{fancyhdr} % header
\usepackage{amsmath} % math formulas
\usepackage{index}
\usepackage[backend=biber, sorting=none]{biblatex}
\addbibresource{library.bib}
\usepackage[portuguese]{babel}
\usepackage[paper=a4paper, top=2.5cm, bottom=2.5cm, left=2.5cm, right=2.5cm]{geometry}
%\usepackage{indentfirst}
\usepackage{bm}
\newcommand{\bfitDelta}{\bm{\mathit{\Delta}}}
\usepackage{caption}
% \numberwithin{equation}{section}
\usepackage{blindtext}

\newcommand\tabitem{\setlength{\itemindent}{1cm}}
\newcommand\tabtabitem{\setlength{\itemindent}{2cm}}

\usepackage{ulem}
\usepackage{times}   % Times New Roman

\pagestyle{fancy}
\fancyhf{}
\rfoot{\thepage}


\setlength\parindent{24pt}
\newcommand\tab[1][0.8cm]{\hspace*{#1}}

\renewcommand{\headrulewidth}{0pt}
%\newcommand\notab{\setlength{\parindent}{0cm}}
%\textbf{NEGRITO}
%\textit{ITÁLICO}
%\underline{SUBLINHADO}
\usepackage{afterpage}

\usepackage{appendix}

\usepackage{chngcntr}
\counterwithin{figure}{section} % NUMERAÇÃO DAS FIGURAS CONSOANTE CAPÍTULO
\counterwithin{table}{section}  % // TABELAS



\begin{document}

\renewcommand{\listfigurename}{Índice de figuras}

\renewcommand{\listtablename}{Índice de tabelas}
    
%\DeclareNameAlias{sortname}{family-given}
%\DeclareNameAlias{default}{family-given}

\begin{titlepage}
\newgeometry{top=1.5cm, bottom=2cm}


   \begin{center}
        \includegraphics[width=0.3\textwidth]{0 - Capa/EEUMfinal.png}\\
        \vspace{0.2cm}
        \textbf{Licenciatura em Engenharia Eletrónica e de Computadores}
       \vfill
        
       \textbf{\Large{Caloiros De Elite: Space Invaders}}\\
       \vspace{0.2cm}
       \textbf{\large{Complementos de Programação de Computadores}}\\

       \vfill
   

        Ano Letivo de 2021/2022\\
        \vspace{0.2pt}
        a101168, Afonso Gomes\\
        a101170, Christian Garcia\\
        \vspace{0.2pt}
        Guimarães, maio de 2022  \vspace{0.8cm}
   \end{center}        % CAPA
\end{titlepage}     % TÍTULO



\pagebreak




\newgeometry{top=2.5cm, left=2.5cm, right=2.5 cm, bottom=2.5cm}
\pagenumbering{roman}


\setcounter{secnumdepth}{-1}
\section{Resumo}
% \markboth{Resumo}{Resumo}

\tab Este trabalho consiste numa explicação extensa da criação do jogo \textbf{"Caloiros de Elite"}, inspirado no jogo \textbf{"Space Invaders"}, desenvolvido pela \textit{Taito Corporation} \cite{spaceInv}.  \par
\vspace{8pt}

     % 2º parágrafo
     
\vspace{8pt}

     % 3º parágrafo
\vspace{40pt}


\pagebreak

\pagebreak


\renewcommand{\contentsname}{Índice}        % ÍNDICE
\tableofcontents
\addcontentsline{toc}{section}{Índice}

\vspace{40pt}
\listoffigures

\vspace{40pt}
\listoftables

\pagebreak


\setcounter{secnumdepth}{3}


\section{Introdução}\label{Intro}

\pagenumbering{arabic}
\setcounter{page}{1}
\tab 

\vspace{8pt}

Grande parte da população que programa as ferramentas de software que utilizamos na atualidade viu-se motivada pelos videojogos da geração dos 70s, 80s e 90s. PACMAN, Tetris e Space Invaders são alguns exemplos deste fenómeno que cada dia influencia a mais jovens e adultos para se mergulharem no imenso banco de dados que conhecemos como a Internet e iniciarem a programar.

\vspace{8pt}

O objetivo deste trabalho é desenvolver um jogo, inspirado por Space Invaders, utilizando as diferentes ferramentas fornecidas pela linguagem de programação C++, nomeadamente, a utilização de OPP (\textit{Object Oriented Programming}), leitura e alteração de ficheiros, entre outros.

\vspace{8pt}

O trabalho distribui-se em diferentes temáticas, sendo estas a descrição geral do jogo (natureza e estratégias utilizadas para a sua implementação),  arquitetura do sistema (são apresentados , de forma geral, os módulos utilizados e a estrutura do jogo) e finalmente a implementação do mesmo, sendo neste último onde vai-se expor ao pormenor cada uma das soluções utilizadas no presente problema.

\pagebreak

\section{Descrição do Problema}

\pagebreak

\section{Arquitetura do Sistema}
\pagebreak

\section{Implementação do Jogo}
\pagebreak

\subsection{Representação do Estado do Jogo}
\pagebreak

\subsection{Inicialização do Estado do Jogo}
\pagebreak

\subsection{Visualização do Estado do Jogo}
\pagebreak

\subsection{Mudança de estado/movimento do Jogador}
\pagebreak

\subsection{Salvar e Restaurar Jogo}
\pagebreak

\subsection{Final da Execução}
\pagebreak

\tab Neste sub capítulo serão abordadas figuras, tabelas, equações e listas numeradas/não numeradas.

\subsubsection{Figuras}

\tab A figura \ref{fig:exemplo} é um exemplo da introdução de figuras em \LaTeX:

\begin{figure}[!ht]
    \centering
    \caption{Legenda da figura}
    \label{fig:exemplo}
\end{figure}


\subsubsection{Tabelas}

\tab Na tabela \ref{tab:exemplo} apresenta-se um exemplo de uma tabela em \LaTeX:

\begin{table}[htb]
\centering
\caption{abcde \cite{exemplo1, exemplo2}}
\label{tab:exemplo}
\begin{tabular}{|c|c|}
\hline
\textbf{a}                                   & \textbf{b}               \\ \hline
a        & b       \\ \hline
a   & b $\approx$ 0,125 $m^{3}$     \\ \hline
a & b     \\ \hline
c        & d \\ \hline
\end{tabular}
\end{table}

\subsubsection{Equações}

\tab Na equação \ref{eq:conservaçãomassaenergia} apresenta-se um exemplo de uma tabela em \LaTeX:

\begin{equation}\label{eq:conservaçãomassaenergia}
    E = m \cdot c^{2}
\end{equation}

\subsubsection{Listas}

\tab Apresente-se agora um exemplo de uma lista numerada:

\begin{enumerate}
    \tabitem
    \item item 1;
    \item item 2.
\end{enumerate}

Relativamente a listas não numeradas:

\begin{itemize}
    \tabitem
    \item item 1;
    \item item 2.
\end{itemize}

\pagebreak


\section{Conclusões e Perspetivas do Trabalho}

\tab \blindtext


\pagebreak






\printbibliography[title={Referências Bibliográficas}]


\pagebreak





\appendix

\section{Bibliografia}

\tab \Blindtext

\pagebreak

\end{document}
